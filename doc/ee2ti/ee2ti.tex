\documentclass[aps,twocolumn,superscriptaddress]{revtex4-1}
\usepackage{color}
\usepackage{graphicx}
\usepackage{epstopdf}
\epstopdfsetup{update}
\usepackage{amsmath}
\usepackage{amssymb}
\usepackage[colorlinks,linkcolor=blue,anchorcolor=blue,citecolor=blue,urlcolor=blue]{hyperref}

\newcommand{\bea}{\begin{eqnarray}}
\newcommand{\eea}{\end{eqnarray}}
\newcommand{\bF}{\mathbf{F}}
\newcommand{\ba}{\mathbf{a}}
\newcommand{\bA}{\mathbf{A}}
\newcommand{\bu}{\mathbf{u}}
\newcommand{\bg}{\mathbf{g}}
\newcommand{\bB}{\mathbf{B}}
\newcommand{\bd}{\mathbf{d}}
\newcommand{\be}{\mathbf{e}}
\newcommand{\bbm}{\mathbf{m}}
\newcommand{\bM}{\mathbf{M}}
\newcommand{\bv}{\mathbf{v}}
\newcommand{\bV}{\mathbf{V}}
\newcommand{\bp}{\mathbf{p}}
\newcommand{\bP}{\mathbf{P}}
\newcommand{\br}{\mathbf{r}}
\newcommand{\bx}{\mathbf{x}}
\newcommand{\bR}{\mathbf{R}}
\newcommand{\bN}{\mathbf{N}}
\newcommand{\bbell}{\boldsymbol{\ell}}
\newcommand{\bL}{\mathbf{L}}
\newcommand{\btau}{\boldsymbol{\tau}}
\newcommand{\bT}{\mathbf{T}}
\newcommand{\bi}{\mathbf{i}}
\newcommand{\bj}{\mathbf{j}}
\newcommand{\bk}{\mathbf{k}}
\newcommand{\bn}{\mathbf{n}}
\newcommand{\bomega}{\boldsymbol{\omega}}
\newcommand{\md}{\mathrm{d}}
\newcommand{\ie}{\textit{i.e.{ }}}
\newcommand{\eg}{\textit{e.g.{ }}}
\newcommand{\etc}{\textit{etc.{ }}}
\newcommand{\etal}{\textit{et al.{ }}}
\newcommand{\ddt}{\frac{\md}{\md t}}
\newcommand{\ddtt}{\frac{\md^2}{\md t^2}}
\newcommand{\ppt}{\frac{\partial}{\partial t}}
\newcommand{\pptt}{\frac{\partial^2}{\partial t^2}}
\newcommand{\me}{\mathrm{e}}


\begin{document}

\title{Multipartite entanglement among corner states in two dimensional second order topological insulators}
\author{Qiang Wang}
\affiliation{National Laboratory of Solid State Microstructures $\&$ School of Physics, Nanjing University, Nanjing, 210093, China}
\author{Da Wang}\email{dawang@nju.edu.cn}
\affiliation{National Laboratory of Solid State Microstructures $\&$ School of Physics, Nanjing University, Nanjing, 210093, China}
\author{Qiang-Hua Wang}\email{qhwang@nju.edu.cn}
\affiliation{National Laboratory of Solid State Microstructures $\&$ School of Physics, Nanjing University, Nanjing, 210093, China}
\affiliation{Collaborative Innovation Center of Advanced Microstructures, Nanjing University, Nanjing 210093, China}

\begin{abstract}
  In a $d$ dimensional topological insulator of order $d$, there are zero energy states on its corners.
  When the system size is finite, these corner states are coupled together by long range hybridizations among them 
  forming a fully entanged multipartite state. In this work, we propose a scheme to calculate the multipartite
  entanglement entropy on a square lattice, which is well described by a four-point toy model and thus can further 
  identify the high order topological insulator.
\end{abstract}
\maketitle
\section{Introduction}
% close relation between entanglement and topological insulators
% high order topo, corner states, nested entanglement spectrum, corner contribution
% area law, topological entanglement entropy of long range entanglement, our scheme
Entanglement and topological states have close relationship, which is extensively studied in recent years.
\cite{Zeng2015,Laflorencie2017}
In general, a nontrivial entanglement structure, \eg degeneracy of the many body entanglement spectra (ES) or gapless
mode of the single particle ES, always indicates a nontrivial topological state, and vice versa.
\cite{Ryu2006, Fidkowski2010}
The key point is the entanglement boundary in some sense mimics the physical boundary which further reflects the
topological property via the bulk boundary correspondence, \ie a $d$ dimensional topological insulator with gapless
states on its $d-1$ dimensional boundaries. 
%although they may differ in special cases. 
Furthermore, on a fnite size topological insulator with open boundary condition, its zero energy boundary states are
coupled together by long range hybridizations and thus also contribute to the total entanglement entropy, 
which can be used to identify the topological degeneracy induced by boundaries. \cite{Wang2015}

Very recently, high order topological insulators are theoretically predicted and observed in experiments.
\cite{}
Different from conventional topological insulators, the gapless boundary states exist on the $d-n$ dimensional
boundaries for a $n$-th order topological insulator. In special, when $n=d$, the gapless boundary states only exist on
the corners. The usual bipartite entanglement spectra can not be directly applied to identify the high order 
topological insulators unless the entanglement boundary has non-smooth corners, called nested entanglement spectra
proposed by xxx.\cite{}
In this work, we focus on the case with open boundary condition, in which case the corner states are coupled together by
long range hybridizations among them to form a multipartite entangled state. We propose a scheme to extract the 
multipartite entanglement entropy from the calculations on the whole lattice. We obtain a universal value of the 
quadripartite entanglement entropy which is well described by a toy model with only four sites and and thus directly 
identify the existence of the gapless corner states. 



\section{Model}

\section{Multi-partite entanglement entropy}

\section{Toy model}
In this section, we use a toy model including only four corner states to explain the numerically obtained universal
value of the quadripartite entanglement entropy. Soving the four-point tight binding model with only nearest hopping
together with a small flux (to break the degeneracy), its two occupied states are obtained
\begin{eqnarray}
  |\Psi_1\rangle&=&\frac12\left(|1\rangle-|2\rangle+|3\rangle-|4\rangle\right) \\
  |\Psi_2\rangle&=&\frac12\left(|1\rangle-i|2\rangle-|3\rangle+i|4\rangle\right) 
\end{eqnarray}
The total ground state is then given by $\Psi\rangle=|\Psi_1\rangle \times |\Psi_2\rangle$, which is a fully entangled
quadripartite state in space\footnote{In Fork space, it is a direct product state as a result of the non interacting
property}. Then the eigenvalues of the reduced correlation matrices can be easily solved out as
\begin{eqnarray}
  \lambda_{12}=\lambda_{14}=\left\{\frac{1}{2}+\frac{\sqrt{2}}{4},\frac{1}{2}-\frac{\sqrt{2}}{4}\right\}
\end{eqnarray}
and
\begin{eqnarray}
  \lambda_{13}=\left\{\frac{1}{2},\frac{1}{2}\right\}
\end{eqnarray}
which gives the quadripartite entanglement entropy
\begin{eqnarray}
  \Delta S= 4\ln2-\sqrt{2}\ln\frac{2+\sqrt{2}}{2-\sqrt{2}}=0.2797
\end{eqnarray}
The value is exactly the same as what we obtained in numerics by subtracting the result of $L_y\gg L_x$ 
from the square case with $L_y=L_x$ as $Lx\rightarrow\infty$. 
In contrast, in the toy model by setting $L_x\rightarrow\infty$, \ie cutting off the hoppings between sites $14$ and
$23$. Then, the gound state is a product state
\begin{eqnarray}
  |\Psi\rangle=|\Psi_{14}\rangle \times |\Psi_{23}\rangle
\end{eqnarray}
which gives $\Delta S=0$ clearly. Comparing with the numerical results on full lattice, it becomes clear that the
nonzero $\Delta S$ must come from the contribution from bulk states near the non-smooth corners of the entanglement 
boundary \cite{Laflorencie2016} similar to the nested entanglement spectrum\cite{Schindler2017}. 

\section{Conclusion and discussion}

\section{acknowledgement}
We thank xxx. This work is supported by NSFC (Nos. xxxx and yyyy).

\bibliography{ee2ti}
\end{document}
