\documentclass[aps,twocolumn,superscriptaddress]{revtex4-1}
\usepackage{color}
\usepackage{graphicx}
\usepackage{epstopdf}
\epstopdfsetup{update}
\usepackage{amsmath}
\usepackage{mathtools}
\usepackage[colorlinks,linkcolor=blue,anchorcolor=blue,citecolor=blue,urlcolor=blue]{hyperref}
\newcommand{\bea}{\begin{eqnarray}}
\newcommand{\eea}{\end{eqnarray}}
\newcommand{\bF}{\mathbf{F}}
\newcommand{\ba}{\mathbf{a}}
\newcommand{\bA}{\mathbf{A}}
\newcommand{\bu}{\mathbf{u}}
\newcommand{\bg}{\mathbf{g}}
\newcommand{\bB}{\mathbf{B}}
\newcommand{\bd}{\mathbf{d}}
\newcommand{\be}{\mathbf{e}}
%\newcommand{\bm}{\mathbf{m}}
\newcommand{\bM}{\mathbf{M}}
\newcommand{\bv}{\mathbf{v}}
\newcommand{\bV}{\mathbf{V}}
\newcommand{\bp}{\mathbf{p}}
\newcommand{\bq}{\mathbf{q}}
\newcommand{\bP}{\mathbf{P}}
\newcommand{\br}{\mathbf{r}}
\newcommand{\bx}{\mathbf{x}}
\newcommand{\bR}{\mathbf{R}}
\newcommand{\bN}{\mathbf{N}}
%\newcommand{\bell}{\boldsymbol{\ell}}
\newcommand{\bL}{\mathbf{L}}
\newcommand{\btau}{\boldsymbol{\tau}}
\newcommand{\bT}{\mathbf{T}}
\newcommand{\bi}{\mathbf{i}}
\newcommand{\bj}{\mathbf{j}}
\newcommand{\bk}{\mathbf{k}}
\newcommand{\bn}{\mathbf{n}}
\newcommand{\bomega}{\boldsymbol{\omega}}
\newcommand{\md}{\mathrm{d}}
\newcommand{\ie}{\textit{i.e.{ }}}
\newcommand{\eg}{\textit{e.g.{ }}}
\newcommand{\etc}{\textit{etc.{ }}}
\newcommand{\ddt}{\frac{\md}{\md t}}
\newcommand{\ddtt}{\frac{\md^2}{\md t^2}}
\newcommand{\ppt}{\frac{\partial}{\partial t}}
\newcommand{\pptt}{\frac{\partial^2}{\partial t^2}}
\newcommand{\me}{\mathrm{e}}


%===============================================
\begin{document}
\title{Critical exponents of nonlinear sigma model on Grassmann manifold $U(N)/U(m)U(N-m)$ by $1/N$ expansion}
\author{Shan-Yue Wang}
\affiliation{National Laboratory of Solid State Microstructures $\&$ School of Physics, Nanjing University, Nanjing, 210093, China}
\author{Da Wang}\email{dawang@nju.edu.cn}
\affiliation{National Laboratory of Solid State Microstructures $\&$ School of Physics, Nanjing University, Nanjing, 210093, China}
\author{Congjun Wu}\email{cjwu@physics.ucsd.edu}
\affiliation{UCSD}
\author{Qiang-Hua Wang}\email{qhwang@nju.edu.cn}
\affiliation{National Laboratory of Solid State Microstructures $\&$ School of Physics, Nanjing University, Nanjing, 210093, China}
\affiliation{Collaborative Innovation Center of Advanced Microstructures, Nanjing University, Nanjing 210093, China}
\begin{abstract}
  Motivated by the numerical evidence of a continuous phase transition between Neel and paramagnetic phases in the SU(N) Hubbbard model, we studied its low energy nonlinear sigma model defined on Grassman manifold $U(N)/U(m)U(N-m)$ using the complex projective presentation, which is a direct generalization of the widely studied CP$^{N-1}$ model (corresponding to $m=1$). In space-time dimension $2<d<4$, to the first order of $1/N$, we calculate the critical exponents of the Neel moment, which are all functions of $m/N$ indicating that larger $m$ effectively reduces $N$ and thus brings stronger fluctuations around the $N=\infty$ fixed point. 
\end{abstract}
\maketitle

\section{Introduction}
% SU(N)-spin symmetry breaking: dependence of m
% physical realization: cold atom fermionic system
% universality class, CP^{N-1} model
Spin is a fundamental property of elementary particles inherbted from the spatial SO(3) symmetry. For a single electron, there are only three indepedent spin directions corresponding to three SO(3) or SU(2) generators. In solids, different electrons can be combined by Hund's rule to form larger spins as different representations of the SO(3) or SU(2) group. If all other degrees of freedom (\eg charge) are frozen, the system is said to be described by an effective spin model, \eg the famous Heisenberg model. Due to the lacking of small parameters, the spin-$1/2$ Heisenberg model is difficult to solve by perturbations. In such a backgound, people introduced the SU(N) spin such that $1/N$ can now be taken as a perturbation parameter. One important consequence of the larger SU(N) group is it has more ($N^2-1$) independent spin directions (generators), which can be broken into different kinds of magnetic ordered states. Taking antiferromagnetic order as an example, the spin in A-sublattice is represented by an $m$ rows and $n_c$ columns Yougn tableau, while the spin in B-sublattice belong to its conjugate representation, \ie $N-m$ rows and $n_c$ columns. In recent years, the SU(N) Heisenberg model and its variations have been extensively studied theoretically mainly focusing on the case of $m=1$. On the other hand, such an SU(N) spin system can now be simulated in cold atom experiments since the carriers in an optical lattice is not electrons but atoms with larger hyperfine nuclear spins spanning the internal space. 

\section{Model}
% from Hubbard model to SU(N)-AF representations, non linear sigma model and then derive the CP model
The Neel order parameter of the SU(N) Hubbard model is defined by \bea \mathcal{N}_i=(-1)^i\langle \Psi_i^\dagger \sigma_b \Psi_i\rangle \eea where $\Psi_i$ is an N-flavor fermion field and $\sigma_b$ ($1\le b\le N^2-1$) denote the SU(N) group generators with normalization condition $\mathrm{tr}(\sigma_b^2)=N$ during this work. In the normal state all these $N^2-1$ branches of magnetic modes degenerate indicating the SU(N) symmetry. While in the Neel state, the SU(N) symmetry can be broken into different irreducible representations which are characterized by an additional parameter $m$. These irreducible representations can be described by Young tableau with $m$ and $(N-m)$ vertical boxes on A- and B-sublattice. In the language of fermions in our case, it means $m$ and $(N-m)$ fermions on two kinds of sublattice, respectively. For simplicity but without loss of generality, we can always use a diagonal matrix \bea \sigma_b=\text{diag}\left[P_+,\cdots,P_+,-P_,\cdots,-P_-\right] \eea to represent the generator where $P_+=P_-^{-1}=\sqrt{(N-m)/m}$, and thus the order parameter is written as \bea \mathcal{N}_i=(-1)^i\left[P_+\sum_{\alpha=1}^m n_{i\alpha}-P_-\sum_{\alpha=m+1}^{N}n_{i\alpha} \right]. \eea where $n_{i\alpha}$ is the fermion number with flavor $\alpha$. 

Near the AF-PM phase transition of the SU(N) Hubbard model, fermionic excitations are gapped out and thus the critical behavior is governed by the nonlinear sigma model
\bea \label{eq:NLsM}S=\frac{\rho_s}{2}\int \partial_\mu\bn \cdot \partial_\mu\bn \eea
where $\rho_s$ is the spin stiffness, $\mu$ denotes the space-time indices, $\int$ means integral over space dimensions and $\bn$ is the normalized Neel moment $\mathcal{N}$. The action $S$ is directly related to the partition function $Z=\int \me^{-S}$. In this work, we parametrize $\bn$ using the complex representation \bea \label{eq:cprep}n_b=\sum_{i=1}^m z_i^\dag \sigma_b z_i=\mathrm{Tr}(Z^\dag\sigma_b Z) \eea where $z_i$ is an N-flavor boson field and $m$ copies of $z_i$ are put together to form a complex $N\times m$ matrix $Z$. Next, applying the Fierz identity of the $SU(N)$ group
\bea \label{eq:Fierz} (\sigma_b)_{\alpha\beta} (\sigma_b)_{\gamma\delta}+\delta_{\alpha\beta}\delta_{\gamma\delta}=N\delta_{\alpha\delta}\delta_{\beta\gamma} \eea
we get
\bea \bn\cdot\bn = N\mathrm{Tr}(ZZ^\dag Z Z^\dag)-\mathrm{Tr}(ZZ^\dag)^2 \eea
In order to maintain $\bn\cdot\bn=1$, we impose a constraint condition \bea Z^\dag Z=\frac{1}{\sqrt{m(N-m)}}I \label{eq:normalizeZ}\eea
which indicates that the $Z$ field in fact lives on the Grassmann manifold $U(N)/U(m)U(N-m)$. 

Substituting the complex representation Eq.~\ref{eq:cprep} into Eq.~\ref{eq:NLsM} and again applying Fierz identity Eq.~\ref{eq:Fierz}, we have
\bea S &=& N\rho_s \int\mathrm{Tr}\left[-(i\partial_\mu Z^\dagger Z) (-iZ^\dag \partial_\mu Z) \right. \nonumber\\  &+& \left. Z^\dag Z (i\partial_\mu Z^\dagger)(-i\partial_\mu Z)\right] .\eea
The first term can be decoupled by introducing a Hubbard-Stratonovich field $\bA$, 
\bea S=N\rho_s\int \mathrm{Tr}\left[ (i\partial_\mu Z^\dag+A_\mu Z^\dag)(-i\partial_\mu Z+ZA_\mu)  \right] \eea
Notice here $A_\mu$ is an $m\times m$ matrix and thus resembles a non Abelian gauge field.
The $Z$ field can be further rescaled to eliminate the prefactor $N\rho_s$, which now appears in the right hand side of the constraint condition Eq.~\ref{eq:normalizeZ}. Using the rescaled field and introducing a real Lagrangian multiplier matrix $\lambda$ to incorporate the constraint condition, we arrive at
\bea \label{eq:cpNm} S&=&\int \mathrm{Tr}\left[ (i\partial_\mu Z^\dag+A_\mu Z^\dag)(-i\partial_\mu Z+ZA_\mu)  \right] \nonumber \\ &+& i\int \mathrm{Tr}\left\{\lambda\left[Z^\dag Z-\frac{N\rho_s}{\sqrt{m(N-m)}}I\right]\right\} . \eea 
This model is a direct generalization of the famous CP$^{N-1}$ model \cite{} (special case with $m=1$). Their main change of $m>1$ relative to $m=1$ is: all $Z$, $A$, $\lambda$ fields now become matrices and thus make the model more complex. Such a model was first proposed by MacFarlane \cite{} and latter studied by some other researchers \cite{Hikomi, Duerksen, Read&Sachdev}. However, due to the missing of a real physical system governed by such a model, it received less and less attentions in recent years. In this work, we rediscover and apply it to study the critical behaviors in the AF-PF phase transition in the SU(N) symmetric system. We will only consider the renormalized classical region (keeping only the smallest Matsubara frequency) since it is sufficient to obtain the critical exponents. \cite{IKK}

\section{Large-N limit}
We first study the large N limit, in which case the saddle point solution becomes exact as a result of the global prefactor $N$ after tracing out the $Z$-field. The saddle point condition $\delta S/\delta bA_\mu$ gives $A_\mu=0$ which means there is no gauge field. While the other saddle point condition $\delta S/\delta \lambda=0$ gives just the constraint condition. In order to describe the ordered phase, we assume 
\bea Z_{\alpha i}=z_0\delta_{\alpha i}+z_{\alpha i}. \eea 
where $\alpha\le N$ and $i\le m$. The first term means condensation occurs when $z_0\ne0$ with a remaining SU(m) symmetry and the second term describes fluctuations. Now the constraint condition becomes
\bea \label{eq:Ninfty} z_0^2 + NT\int \frac{1}{k^2} = \frac{N\rho_s}{\sqrt{m(N-m)}} \eea
It is interesting to note that $z_0^2$ scaled to $N$ which is a direct consequence of the rescaling operation before Eq.~\ref{eq:cpNm}. 

We first examine the critical exponents of the $Z$ field. At $T=T_c$, $z_0=0$ and $G_z(k)=k^{-2}$ which gives $\eta_z=0$ by definition $G_z(k)\sim k^{\eta-2}$. The exponent $\gamma_z$ is related to the uniform static susceptibility, or equivalently $G_z(k=0)$ above $T_c$. In this case, a mass term $\Delta$ should be added to retain the constraint condition, \ie $G_z(k)=(k^2+\Delta)^{-1}$. Then from Eq.~\ref{eq:Ninfty}, we get $\Delta\propto t^{2/(d-2)}$ where $t=(T-T_c)/T_c$. Comparing with the definition $G_z(k=0)\sim t^{-\gamma_z}$, we obtain $\gamma_z=2/(d-2)$. From $\eta_z$ and $\gamma_z$, all other exponents of the $Z$ fields can be obtained from scaling relations. In particular, $\nu_z=1/(d-2)$ which should be independent on specified fields and thus we can drop its subscript $z$, \ie $\nu=1/(d-2)$. (See the conclusion section for more discussions about this.)

Next, let us focus on the magnetic moment as a particle-hole composition of the $Z$ field. The order parameter $M$ is determined by the condensed $Z$ field by definition, \ie $M=mP_+z_0^2$. From Eq.~\ref{eq:Ninfty} by setting $T<T_c$ and $T=T_c$, we have $z_0^2\propto t$ and thus $M\propto t$. As a result, the exponent $\beta_n=1$ by comparing with the definition $M\propto t^{\beta_n}$. Then, together with $\nu=1/(d-2)$ obtained from above, all remaining exponents are ready, \eg $\eta_n=d-2$ and $\gamma_n=(4-d)/(d-2)$. 


\section{1/N expansion}
Next, let us go beyond the saddle point approximation by performing a $1/N$ expansion calculation. Our strategy mainly follows the work by IKK1996 which only considered the case with $m=1$. Up to $1/N$ order, both $i\lambda$ and $A$ fields acquire dynamics from their vacuum polarizations,
\bea \Pi_{ij}^\lambda(q)&=&2z_0^2\frac{1}{q^2} + NT\int_k \frac{1}{k^2(k+q)^2} \nonumber\\ &=&\frac{2z_0^2}{q^2}+\frac{TNK_dA_d}{q^{4-d}} \equiv \Pi_\lambda(q) \eea 
and 
\bea \Pi_{ij}^{\mu\nu}(q) &=& 2z_0^2 \frac{q_\mu q_\nu}{q^2} + NT\int \frac{(2k_\mu+q_\mu)(2k_\nu+q_\nu)}{k^2(k+q)^2} \nonumber\\ && - 2z_0^2 \delta_{\mu\nu} -2NT\int \frac{1}{k^2}\delta_{\mu\nu}  \nonumber\\ &=&\left(2z_0^2+\frac{TNK_dA_d}{d-1} q^{d-2}\right)\left(\frac{q_\mu q_\nu}{q^2}-\delta_{\mu\nu}\right) \nonumber\\
&\equiv& \Pi_A(q)\left(\frac{q_\mu q_\nu}{q^2}-\delta_{\mu\nu}\right) \eea
where 
\bea K_d^{-1}=2^{d-1}\pi^{d/2}\Gamma\left(\frac{d}{2}\right) \eea
and
\bea A_d=\frac{\Gamma(d-2)\Gamma(2-d/2)\Gamma^2(d/2-1)}{2\Gamma(d-2)} \eea
One good property is that both $\Pi_{ij}^\lambda(q)$ and $\Pi_{ij}^{\mu\nu}(q)$ show no dependence on their subscript $ij$ and even $m$, which will greatly simplify the following calculations. From the vacuum polarizations, we get their propagators $D_\lambda(q)=-\Pi_\lambda^{-1}$ and
$D_A^{\mu\nu}(q)=-\Pi_A^{-1}\left(\frac{q_\mu q_\nu}{q^2}-\delta_{\mu\nu}\right)$ under the Landau gauge. {\color{red} @WSY: if you like, you can add the gauge fixing parameter $\xi$ explicitly.}

The self energy of the $z$ field is $\Sigma=\Sigma_\lambda+\Sigma_A$ where
\bea \Sigma_\lambda(k)=-mT\int_q \left[\frac{1}{(k+q)^2}-\frac{1}{q^2}\right]\Pi_\lambda^{-1}(q) \eea
and
\bea \Sigma_A(k) &=& mT\int_q \frac{1}{(k+q)^2} D_A^{\mu\nu}(q) (2k_\mu+q_\mu) (2k_\nu+q_\nu) \nonumber \\&=& 4mT\int_q \frac{1}{(k+q)^2}\Pi_A^{-1}\left[ k^2-\frac{(k\cdot q)^2}{q^2} \right] \eea
After completing the bubble integrals, and extracting the coefficient of the $k^2\ln k$ term, (see details in Appendix) we immediately get $\eta_z=-\frac{(4d-7)m}{NA_d}$, which was first obtained by Hikami(1980). However, his calculation was based on single particle propagators and thus can only gave the critical exponents of the $Z$ field. On the other hand, here, we are more interested in the critical exponents of the physical field $\bn$. To this end, we have to consider the correlation functions of $\bn$, which is a two-particle propagator. Keeping only the uniform term (q=0), the susceptibility of $\bn$ is
\bea \chi_0=m^2P_+^2 z_0^4\left[ 1-  2mT\int_k \frac{\Pi_\lambda^{-1}}{k^4} + 4 \left.\frac{\Sigma(k)}{k^2}\right|_{k\rightarrow0} \right] \eea
which gives the magnetic moment square $M^2$ as a function $z_0$. In order to get $\beta_n$, we still need the relation between $z_0$ and $t$, which can be obtained from the constraint condition up to order of $1/N$,
\bea &&z_0^2 \left[ 1-  mT\int_k \frac{\Pi_\lambda^{-1}}{k^4} + 2 \left.\frac{\Sigma(k)}{k^2}\right|_{k\rightarrow0}\right] + NT\int_k \frac{1}{k^2} \nonumber\\&&+ NT\int_k \frac{\Sigma(k)}{k^4} = \frac{N\rho_s}{\sqrt{m(N-m)}} \eea 
After some lengthy but straightforward algebra (see details in Appendix), we get 
\bea \beta_n=1-\frac{2m(d^2-d+2)}{NA_d} \eea
Taking together with the correlation length exponent $\nu=\nu_z$ which had already obtained by Hikami(1980)
\bea \nu=\frac{1}{d-2}\left[ 1-\frac{2md(d-1)}{NA_d} \right] \eea
we get the remaining critical exponents,
\bea \eta_n=(d-2)\left[1-\frac{8m}{NA_d}\right] \eea
and
\bea \gamma_n=\frac{4-d}{d-2}\left[ 1+\frac{8m(d-2)}{(4-d)NA_d}-\frac{2md(d-1)}{NA_d} \right] \eea
Although the calculation is somewhat tedious, the result is quite clear: we only need replace $1/N$ to $m/N$ comparing with the CP$^{N-1}$ model \cite{IKK1996}. Hence, larger $m$ may spoil the first order perturbation results of $1/N$ expansion. Higher order calculations are in need to obtain more relevant results. 


\section{Conclusion and discussion}
In the conventional theory of critical phenomena, there is only one correlation length $\xi$, and thus all physical quantities share the same $\nu$. Such a situation may change, \textit{e.g.} in the deconfined quantum critical phenomena.



{\color{red} The bad news is: we should take $m/N$ as a small parameter to valid the $1/N$-expansion calculation. Therefore, the method cannot be applied to our interested case with $m=N/2$. Anyway, let us keep going on and just take this work as a generalization of the $m=1$ theory.}

\section{acknowledgement}
We thank Katanin, Irkhin, and xxx. This work is supported by NSFC (Nos. xxxx and yyyy).


\bibliography{cpNm}
\end{document}
