\documentclass[aps,twocolumn,superscriptaddress]{revtex4-1}
\usepackage{color}
\usepackage{graphicx}
\usepackage{epstopdf}
\epstopdfsetup{update}
\usepackage{amsmath}
\usepackage{mathtools}
\usepackage[colorlinks,linkcolor=blue,anchorcolor=blue,citecolor=blue,urlcolor=blue]{hyperref}
\newcommand{\bea}{\begin{eqnarray}}
\newcommand{\eea}{\end{eqnarray}}
\newcommand{\bF}{\mathbf{F}}
\newcommand{\ba}{\mathbf{a}}
\newcommand{\bA}{\mathbf{A}}
\newcommand{\bu}{\mathbf{u}}
\newcommand{\bg}{\mathbf{g}}
\newcommand{\bB}{\mathbf{B}}
\newcommand{\bd}{\mathbf{d}}
\newcommand{\be}{\mathbf{e}}
%\newcommand{\bm}{\mathbf{m}}
\newcommand{\bM}{\mathbf{M}}
\newcommand{\bv}{\mathbf{v}}
\newcommand{\bV}{\mathbf{V}}
\newcommand{\bp}{\mathbf{p}}
\newcommand{\bq}{\mathbf{q}}
\newcommand{\bP}{\mathbf{P}}
\newcommand{\br}{\mathbf{r}}
\newcommand{\bx}{\mathbf{x}}
\newcommand{\bR}{\mathbf{R}}
\newcommand{\bN}{\mathbf{N}}
%\newcommand{\bell}{\boldsymbol{\ell}}
\newcommand{\bL}{\mathbf{L}}
\newcommand{\btau}{\boldsymbol{\tau}}
\newcommand{\bT}{\mathbf{T}}
\newcommand{\bi}{\mathbf{i}}
\newcommand{\bj}{\mathbf{j}}
\newcommand{\bk}{\mathbf{k}}
\newcommand{\bn}{\mathbf{n}}
\newcommand{\bomega}{\boldsymbol{\omega}}
\newcommand{\md}{\mathrm{d}}
\newcommand{\ie}{\textit{i.e.{ }}}
\newcommand{\etc}{\textit{etc.{ }}}
\newcommand{\ddt}{\frac{\md}{\md t}}
\newcommand{\ddtt}{\frac{\md^2}{\md t^2}}
\newcommand{\ppt}{\frac{\partial}{\partial t}}
\newcommand{\pptt}{\frac{\partial^2}{\partial t^2}}
\newcommand{\me}{\mathrm{e}}


%===============================================
\begin{document}
\title{Critical exponents of nonlinear sigma model on Grassmann manifold $U(N)/U(m)U(N-m)$ by $1/N$ expansion}
\author{Shan-Yue Wang}
\affiliation{National Laboratory of Solid State Microstructures $\&$ School of Physics, Nanjing University, Nanjing, 210093, China}
\author{Da Wang}\email{dawang@nju.edu.cn}
\affiliation{National Laboratory of Solid State Microstructures $\&$ School of Physics, Nanjing University, Nanjing, 210093, China}
\author{Qiang-Hua Wang}\email{qhwang@nju.edu.cn}
\affiliation{National Laboratory of Solid State Microstructures $\&$ School of Physics, Nanjing University, Nanjing, 210093, China}
\affiliation{another address}
\begin{abstract}
  Motivated by the numerical evidence of a continuous phase transition between Neel and paramagnetic phases in the SU(N) Hubbbard model, we studied its low energy nonlinear sigma model defined on Grassman manifold $U(N)/U(m)U(N-m)$ using the complex projective presentation, which is a direct generalization of the widely studied CP$^{N-1}$ model (corresponding to $m=1$). In space-time dimension $2<d<4$, to the first order of $1/N$, we calculate the critical exponents of the Neel moment, which are all functions of $m/N$ indicating that larger $m$ effectively reduces $N$ and thus brings stronger fluctuations around the $N=\infty$ fixed point. 
\end{abstract}
\maketitle

\section{Introduction}

\section{Model}
% from Hubbard model to SU(N)-AF representations, non linear sigma model and then derive the CP model
The Neel order parameter of the SU(N) Hubbard model is defined by \bea \mathcal{N}_i=(-1)^i\langle \Psi_i^\dagger \sigma_b \Psi_i\rangle \eea where $\Psi_i$ is an N-flavor fermion field and $\sigma_b$ ($1\le b\le N^2-1$) denote the SU(N) group generators with normalization condition $\mathrm{tr}(\sigma_b^2)=N$ during this work. In the normal state all these $N^2-1$ branches of magnetic modes degenerate indicating the SU(N) symmetry. While in the Neel state, the SU(N) symmetry can be broken into different irreducible representations which are characterized by an additional parameter $m$. These irreducible representations can be described by Young tableau with $m$ and $(N-m)$ vertical boxes on A- and B-sublattice. In the language of fermions in our case, it means $m$ and $(N-m)$ fermions on two kinds of sublattice, respectively. For simplicity but without loss of generality, we can always use a diagonal matrix \bea \sigma_b=\text{diag}\left[P_+,\cdots,P_+,-P_,\cdots,-P_-\right] \eea to represent the generator where $P_+=P_-^{-1}=\sqrt{(N-m)/m}$, and thus the order parameter is written as \bea \mathcal{N}_i=(-1)^i\left[P_+\sum_{\alpha=1}^m n_{i\alpha}-P_-\sum_{\alpha=m+1}^{N}n_{i\alpha} \right]. \eea where $n_{i\alpha}$ is the fermion number with flavor $\alpha$. 

Near the AF-PM phase transition of the SU(N) Hubbard model, fermionic excitations are gapped out and thus the critical behavior is governed by the nonlinear sigma model
\bea \label{eq:NLsM}S=\frac{\rho_s}{2}\int \partial_\mu\bn \cdot \partial_\mu\bn \eea
where $\rho_s$ is the spin stiffness, $\mu$ denotes the space-time indices, $\int$ means integral over spacetime and $\bn$ is the normalized Neel moment $\mathcal{N}$. The action $S$ is directly related to the partition function $Z=\int \me^{-S}$. In this work, we parametrize $\bn$ using the complex representation \bea \label{eq:cprep}n_b=\sum_{i=1}^m z_i^\dag \sigma_b z_i=\mathrm{Tr}(Z^\dag\sigma_b Z) \eea where $z_i$ is an N-flavor boson field and $m$ copies of $z_i$ are put together to form a complex $N\times m$ matrix $Z$. Next, applying the Fierz identity of the $SU(N)$ group
\bea \label{eq:Fierz} (\sigma_b)_{\alpha\beta} (\sigma_b)_{\gamma\delta}+\delta_{\alpha\beta}\delta_{\gamma\delta}=N\delta_{\alpha\delta}\delta_{\beta\gamma} \eea
we get
\bea \bn\cdot\bn = N\mathrm{Tr}(ZZ^\dag Z Z^\dag)-\mathrm{Tr}(ZZ^\dag)^2 \eea
In order to maintain $\bn\cdot\bn=1$, we impose a constraint condition \bea Z^\dag Z=\frac{1}{\sqrt{m(N-m)}}I \label{eq:normalizeZ}\eea
which indicates that the $Z$ field in fact lives on the Grassmann manifold $U(N)/U(m)U(N-m)$. 

Substituting the complex representation Eq.~\ref{eq:cprep} into Eq.~\ref{eq:NLsM} and again applying Fierz identity Eq.~\ref{eq:Fierz}, we have
\bea S &=& N\rho_s \int\mathrm{Tr}\left[-(i\partial_\mu Z^\dagger Z) (-iZ^\dag \partial_\mu Z) \right. \nonumber\\  &+& \left. Z^\dag Z (i\partial_\mu Z^\dagger)(-i\partial_\mu Z)\right] .\eea
The first term can be decoupled by introducing a Hubbard-Stratonovich field $\bA$, 
\bea S=N\rho_s\int \mathrm{Tr}\left[ (i\partial_\mu Z^\dag+A_\mu Z^\dag)(-i\partial_\mu Z+ZA_\mu)  \right] \eea
Notice here $A_\mu$ is an $m\times m$ matrix and thus resembles a non Abelian gauge field.
The $Z$ field can be further rescaled to eliminate the prefactor $N\rho_s$, which now appears in the right hand side of the constraint condition Eq.~\ref{eq:normalizeZ}. Using the rescaled field and introducing a real Lagrangian multiplier matrix $\lambda$ to incorporate the constraint condition, we arrive at
\bea \label{eq:cpNm} S&=&\int \mathrm{Tr}\left[ (i\partial_\mu Z^\dag+A_\mu Z^\dag)(-i\partial_\mu Z+ZA_\mu)  \right] \nonumber \\ &+& i\int \mathrm{Tr}\left\{\lambda\left[Z^\dag Z-\frac{N\rho_s}{\sqrt{m(N-m)}}I\right]\right\} . \eea 
This model is a direct generalization of the famous CP$^{N-1}$ model \cite{} (special case with $m=1$). Their main change of $m>1$ relative to $m=1$ is: all $Z$, $A$, $\lambda$ fields now become matrices and thus make the model more complex. Such a model was first proposed by MacFarlane \cite{} and latter studied by some other researchers \cite{Hikomi, Duerksen, Read&Sachdev}. However, due to the missing of a real physical system governed by such a model, it received less and less attentions in recent years. In this work, we rediscover and apply it to study the critical behaviors in the AF-PF phase transition in the SU(N) symmetric system. We will only consider the renormalized classical region (keeping only the smallest Matsubara frequency) since it is sufficient to obtain the critical exponents. \cite{IKK}

\section{Large-N limit}
We first study the large N limit, in which case the saddle point solution becomes exact as a result of the global prefactor $N$ after tracing out the $Z$-field. The saddle point condition $\delta S/\delta bA_\mu$ gives $A_\mu=0$ which means there is no gauge field. While the other saddle point condition $\delta S/\delta \lambda=0$ gives just the constraint condition. In order to describe the ordered phase, we assume 
\bea Z_{\alpha i}=z_0\delta_{\alpha i}+z_{\alpha i}. \eea 
where $\alpha\le N$ and $i\le m$. The first term means condensation occurs when $z_0\ne0$ with a remaining SU(m) symmetry and the second term describes fluctuations. Now the constraint condition becomes
\bea z_0^2 + NT\int \frac{1}{k^2} = \frac{N\rho_s}{\sqrt{m(N-m)}} \eea
It is interesting to note that $z_0^2$ scaled to $N$ which is a direct consequence of the rescaling operation before Eq.~\ref{eq:cpNm}. 

The anomalous dimension $\eta_z$ of the $Z$ field can be obtained from $G_z(k)\sim k^{\eta-2}$ at $T=T_c$. Since $G_z(k)=k^{-2}$, we get $\eta_z=0$. 


The longitudinal magnetic susceptibility is
\bea \chi_{bb}(q) &=& m(N-m) z_0^4\delta_{q0} + (N-m)\frac{z_0^2}{q^2} \nonumber\\ 
&+& mNT\int \frac{1}{k^2(k+q)^2} \eea




\begin{itemize}
	\item $G(k)\sim k^{\eta-2}$ at $T=T_c$: $g(k)=k^{-2}$ gives $\eta_z=0$, while $\chi(q)\sim q^{d-4}$ gives $\eta_n=d-2$.
	\item $M\sim t^\beta$ (where $t=(T_c-T)/Tc$) below $T_c$: $M=\sqrt{m}z_0^2\sim t$ gives $\beta=1$.
	\item $\gamma$ can be obtained from $g^{-1}(k=0)$ (zero frequency) and $\nu$ can be obtained from $\xi$ (equal time correlation).
	\item In the conventional theory of critical phenomena, there is only one correlation length $\xi$, and thus all physical quantities share the same $\nu$. Such a situation may change, \textit{e.g.} in the deconfined quantum critical phenomena. What about tri-critical point?
\end{itemize}

\section{1/N expansion}

\section{Summary}

\section{acknowledgement}
We thank Katanin, Irkhin, and xxx. This work is supported by NSFC (Nos. xxxx and yyyy).


\bibliography{cpNm}
\end{document}
