\documentclass[11pt]{article}
\usepackage[margin=1in]{geometry}
\usepackage{color}
\usepackage{graphicx}
\usepackage{epstopdf}
\epstopdfsetup{update}
\usepackage{amsmath}
\usepackage{amssymb}
\usepackage[colorlinks]{hyperref}
\usepackage{geometry}
\usepackage[square]{natbib} 

\newcommand{\bea}{\begin{eqnarray}}
\newcommand{\eea}{\end{eqnarray}}
\newcommand{\bF}{\mathbf{F}}
\newcommand{\ba}{\mathbf{a}}
\newcommand{\bA}{\mathbf{A}}
\newcommand{\bu}{\mathbf{u}}
\newcommand{\bg}{\mathbf{g}}
\newcommand{\bB}{\mathbf{B}}
\newcommand{\bd}{\mathbf{d}}
\newcommand{\be}{\mathbf{e}}
\newcommand{\bm}{\mathbf{m}}
\newcommand{\bM}{\mathbf{M}}
\newcommand{\bv}{\mathbf{v}}
\newcommand{\bV}{\mathbf{V}}
\newcommand{\bp}{\mathbf{p}}
\newcommand{\bP}{\mathbf{P}}
\newcommand{\br}{\mathbf{r}}
\newcommand{\bx}{\mathbf{x}}
\newcommand{\bR}{\mathbf{R}}
\newcommand{\bN}{\mathbf{N}}
\newcommand{\bell}{\boldsymbol{\ell}}
\newcommand{\bL}{\mathbf{L}}
\newcommand{\btau}{\boldsymbol{\tau}}
\newcommand{\bT}{\mathbf{T}}
\newcommand{\bi}{\mathbf{i}}
\newcommand{\bj}{\mathbf{j}}
\newcommand{\bk}{\mathbf{k}}
\newcommand{\bn}{\mathbf{n}}
\newcommand{\bomega}{\boldsymbol{\omega}}
\newcommand{\md}{\mathrm{d}}
\newcommand{\ie}{\textit{i.e.{ }}}
\newcommand{\etc}{\textit{etc.{ }}}
\newcommand{\ddt}{\frac{\md}{\md t}}
\newcommand{\ddtt}{\frac{\md^2}{\md t^2}}
\newcommand{\ppt}{\frac{\partial}{\partial t}}
\newcommand{\pptt}{\frac{\partial^2}{\partial t^2}}
\newcommand{\me}{\mathrm{e}}


%===============================================
\begin{document}
\title{Template}
\author{Da Wang}%\thanks{dawang@nju.edu.cn} \\ \\National Laboratory of Solid State Microstructures $\&$ School of Physics, \\ Nanjing University, Nanjing, 210093, China}
%\date{}
\maketitle

%\begin{abstract}
%Here is the abstract.
%\end{abstract}

%\tableofcontents
%=================================================

\bea H=gR_\uparrow^\dag R_\downarrow^\dag L_\downarrow L_\uparrow + h.c. \eea 
\bea Z&=&\int{ \mathcal{D}R\mathcal{D}L \me^{\int{ \bar{R}_\sigma (-\partial\tau)R_\sigma + \bar{L}_\sigma (-\partial\tau)L_\sigma -g(\bar{R}_\uparrow \bar{R}_\downarrow L_\downarrow L_\uparrow + h.c.)} }} \\
      &=&\int{ \mathcal{D}R\mathcal{D}L \me^{\int {\bar{R}_\sigma (-\partial\tau)R_\sigma + \bar{L}_\sigma (-\partial\tau)L_\sigma +g(\bar{R}_\uparrow \bar{R}_\downarrow-\bar{L}_\uparrow \bar{L}_\downarrow)(R_\downarrow R_\uparrow-L_\downarrow L_\uparrow) - g\bar{R}_\uparrow \bar{R}_\downarrow R_\downarrow R_\uparrow -g \bar{L}_\uparrow \bar{L}_\downarrow L_\downarrow L_\uparrow}}}  \\
      &=&\int{ \mathcal{D}R\mathcal{D}L\mathcal{D}\phi \me^{ \int{  \bar{R}_\sigma (-\partial\tau)R_\sigma + \bar{L}_\sigma (-\partial\tau)L_\sigma +[\phi (\bar{R}_\uparrow \bar{R}_\downarrow-\bar{L}_\uparrow \bar{L}_\downarrow) + h.c.]-\frac1g\phi^*\phi  - g\bar{R}_\uparrow \bar{R}_\downarrow R_\downarrow R_\uparrow -g \bar{L}_\uparrow \bar{L}_\downarrow L_\downarrow L_\uparrow }}} \\
      &=&\int \mathcal{D}R\mathcal{D}L\mathcal{D}\phi \sum_k \frac{(-g)^k}{k!}(\bar{R}_\uparrow \bar{R}_\downarrow R_\downarrow R_\uparrow +\bar{L}_\uparrow \bar{L}_\downarrow L_\downarrow L_\uparrow)^k \nonumber\\ 
      &&  \me^{ \int{  \bar{R}_\sigma (-\partial\tau)R_\sigma + \bar{L}_\sigma (-\partial\tau)L_\sigma +[\phi (\bar{R}_\uparrow \bar{R}_\downarrow-\bar{L}_\uparrow \bar{L}_\downarrow) + h.c.]-\frac1g\phi^*\phi  }} \\
      &=&\int \mathcal{D}\phi\me^{\int -\frac{1}{g}\phi^*\phi } \sum_k \frac{(-g)^k}{k!} D_{2k}[\phi] \eea
where $D_{2k}$ is a determent of a rank-$2k$ matrix filled with Green's functions. Such an expression is a convergent series expansion as a result of a fluctuating boson field $\phi$. We can perform a Monte Carlo simulation based on this expansion, called interaction-expansion continuous time quantum Monte Carlo. (Rubtsov {\it et al.} 2006) In this formula, the exact Green's function is given by 
\bea G_R=\frac{1}{Z}\int \mathcal{D}\phi\me^{\int -\frac{1}{g}\phi^*\phi } \sum_k \frac{(-g)^k}{k!} D_{2k+1}[\phi] =\langle \frac{D_{2k+1}}{D_{2k}}\rangle\eea

Let us seek an analytical approximation. A useful approximation is $\phi(\tau)=\phi$, \ie time independent. Then, we have
\bea Z=\int \md \phi \sum_k \frac{(-g)^k}{k!} D_{2k}(\phi) \me^{-\frac{\beta}{g}\phi^*\phi} \eea 
A second approximation is to truncate $k$ with a upper bound $N_k$. 

When $N_k=0$,
%\bea Z=\int \md \phi \me^{-\frac{\beta}{g}\phi^*\phi} \eea

\bea G_R=\frac{1}{Z}\int\md \phi \frac{1}{2}\left(\frac{1}{i\omega_n-|\phi|}+\frac{1}{i\omega_n+|\phi|}\right)\me^{-\beta F(\phi)} \eea
where
\bea F=-2T\log(1+\me^{-\beta|\phi|})-2T\log(1+\me^{\beta|\phi|})+\frac{|\phi|^2}{g} \eea 

Next, let's set $N_k=1$.

\bibliography{bibfile}
%\begin{thebibliography}{99}
%	\bibitem{xxxx} xxxx, PRL 88, 888888 (2088).
%\end{thebibliography}
\end{document}
