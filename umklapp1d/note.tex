\documentclass[12pt]{article}
\usepackage[margin=0.5in]{geometry}
\usepackage{color}
\usepackage{graphicx}
\usepackage{epstopdf}
\epstopdfsetup{update}
\usepackage{amsmath}
\usepackage{amssymb}
\usepackage[colorlinks]{hyperref}
\usepackage{geometry}
\usepackage[square]{natbib} 

\newcommand{\bea}{\begin{eqnarray}}
\newcommand{\eea}{\end{eqnarray}}
\newcommand{\bF}{\mathbf{F}}
\newcommand{\ba}{\mathbf{a}}
\newcommand{\bA}{\mathbf{A}}
\newcommand{\bu}{\mathbf{u}}
\newcommand{\bg}{\mathbf{g}}
\newcommand{\bB}{\mathbf{B}}
\newcommand{\bd}{\mathbf{d}}
\newcommand{\be}{\mathbf{e}}
\newcommand{\bm}{\mathbf{m}}
\newcommand{\bM}{\mathbf{M}}
\newcommand{\bv}{\mathbf{v}}
\newcommand{\bV}{\mathbf{V}}
\newcommand{\bp}{\mathbf{p}}
\newcommand{\bP}{\mathbf{P}}
\newcommand{\br}{\mathbf{r}}
\newcommand{\bx}{\mathbf{x}}
\newcommand{\bR}{\mathbf{R}}
\newcommand{\bN}{\mathbf{N}}
\newcommand{\bell}{\boldsymbol{\ell}}
\newcommand{\bL}{\mathbf{L}}
\newcommand{\btau}{\boldsymbol{\tau}}
\newcommand{\bT}{\mathbf{T}}
\newcommand{\bi}{\mathbf{i}}
\newcommand{\bj}{\mathbf{j}}
\newcommand{\bk}{\mathbf{k}}
\newcommand{\bn}{\mathbf{n}}
\newcommand{\bomega}{\boldsymbol{\omega}}
\newcommand{\md}{\mathrm{d}}
\newcommand{\ie}{\textit{i.e.{ }}}
\newcommand{\etc}{\textit{etc.{ }}}
\newcommand{\ddt}{\frac{\md}{\md t}}
\newcommand{\ddtt}{\frac{\md^2}{\md t^2}}
\newcommand{\ppt}{\frac{\partial}{\partial t}}
\newcommand{\pptt}{\frac{\partial^2}{\partial t^2}}
\newcommand{\me}{\mathrm{e}}


%===============================================
\begin{document}
\title{Two-points umklapp model}
\author{Da Wang}%\thanks{dawang@nju.edu.cn} \\ \\National Laboratory of Solid State Microstructures $\&$ School of Physics, \\ Nanjing University, Nanjing, 210093, China}
\date{}
\maketitle

\begin{abstract}
We are seeking a diagrammatic treatment of the two-points umklapp model.
\end{abstract}

%\tableofcontents
%=================================================
The model is defined as
\bea H=gR_\uparrow^\dag R_\downarrow^\dag L_\downarrow L_\uparrow + h.c. \eea
whose partition function is
\bea Z&=&\int{ \mathcal{D}R\mathcal{D}L \me^{\int{ \bar{R}_\sigma (-\partial\tau)R_\sigma + \bar{L}_\sigma (-\partial\tau)L_\sigma -g(\bar{R}_\uparrow \bar{R}_\downarrow L_\downarrow L_\uparrow + h.c.)} }} \\
&=&\int{ \mathcal{D}R\mathcal{D}L \me^{\int {\bar{R}_\sigma (-\partial\tau)R_\sigma + \bar{L}_\sigma (-\partial\tau)L_\sigma +g(\bar{R}_\uparrow \bar{R}_\downarrow-\bar{L}_\uparrow \bar{L}_\downarrow)(R_\downarrow R_\uparrow-L_\downarrow L_\uparrow) - g\bar{R}_\uparrow \bar{R}_\downarrow R_\downarrow R_\uparrow -g \bar{L}_\uparrow \bar{L}_\downarrow L_\downarrow L_\uparrow}}}  \\
&=&\int{ \mathcal{D}R\mathcal{D}L\mathcal{D}\phi \me^{ \int{  \bar{R}_\sigma (-\partial\tau)R_\sigma + \bar{L}_\sigma (-\partial\tau)L_\sigma +[\phi (\bar{R}_\uparrow \bar{R}_\downarrow-\bar{L}_\uparrow \bar{L}_\downarrow) + h.c.]-\frac1g\phi^*\phi  - g\bar{R}_\uparrow \bar{R}_\downarrow R_\downarrow R_\uparrow -g \bar{L}_\uparrow \bar{L}_\downarrow L_\downarrow L_\uparrow }}} \\
&=&\int \mathcal{D}R\mathcal{D}L\mathcal{D}\phi \sum_k \frac{(-g)^k}{k!} \int \md^k\tau \mathcal{T}_\tau (\bar{R}_\uparrow \bar{R}_\downarrow R_\downarrow R_\uparrow +\bar{L}_\uparrow \bar{L}_\downarrow L_\downarrow L_\uparrow)^k \nonumber\\ 
&&  \me^{ \int{  \bar{R}_\sigma (-\partial\tau)R_\sigma + \bar{L}_\sigma (-\partial\tau)L_\sigma +[\phi (\bar{R}_\uparrow \bar{R}_\downarrow-\bar{L}_\uparrow \bar{L}_\downarrow) + h.c.]-\frac1g\phi^*\phi  }} \\
&\rightarrow&\int \mathcal{D}\phi\sum_k\frac{(-g)^k}{k!}\int \md^k\tau\mathcal{T}_\tau(\partial_{\eta_\uparrow}\partial_{\eta_\downarrow}\partial_{\bar{\eta}_\downarrow}\partial_{\bar{\eta}_\uparrow}+\partial_{\xi_\uparrow}\partial_{\xi_\downarrow}\partial_{\bar{\xi}_\downarrow}\partial_{\bar{\xi}_\uparrow})^k \me^{\mathrm{Tr}\log(\partial_\tau+\hat{\phi})+\mathrm{Tr}\log(\partial_\tau-\hat{\phi})+\bar{\eta}(\partial_\tau+\hat{\phi})\eta+\bar{\xi}(\partial_\tau-\hat{\phi})\xi-\frac1g\phi^*\phi} \\
&=& \int \mathcal{D}\phi \sum_k \frac{(-g)^k}{k!}\int \md^k\tau \mathcal{T}_\tau (\partial_{\eta_\uparrow}\partial_{\eta_\downarrow}\partial_{\bar{\eta}_\downarrow}\partial_{\bar{\eta}_\uparrow}+\partial_{\xi_\uparrow}\partial_{\xi_\downarrow}\partial_{\bar{\xi}_\downarrow}\partial_{\bar{\xi}_\uparrow})^k \me^{S[\phi,\eta,\xi]} \eea
where we have introduced the source $\eta$ and $\xi$ coupled to $R$ and $L$, respectively.
The exact Green's function $G_R$ is given by
\bea \hat{G}_R&=&\frac{1}{Z}\int \mathcal{D}\phi \sum_k \frac{(-g)^k}{k!} \int\md^k\tau\mathcal{T}_\tau \partial_{\bar{\eta}} \partial_\eta (\partial_{\eta_\uparrow}\partial_{\eta_\downarrow}\partial_{\bar{\eta}_\downarrow}\partial_{\bar{\eta}_\uparrow}+\partial_{\xi_\uparrow}\partial_{\xi_\downarrow}\partial_{\bar{\xi}_\downarrow}\partial_{\bar{\xi}_\uparrow})^k \me^{S[\phi,\eta,\xi]} \\
&=&\frac{\int\mathcal{D}\phi \left[\sum {\rm connected}\right]_\phi\left[\sum {\rm closed}\right]_\phi \me^{S[\phi]} }{\int\mathcal{D}\phi \left[\sum {\rm closed}\right]_\phi \me^{S[\phi]}}
\eea
Clearly, if without $\phi$, the closed diagrams can be canceled. But when $\phi$ exists, we cannot do such a simplification unless $|\phi|$ is pinned at a fixed value like in $T=\infty$ and $T=0$ limits. The former is just the usual diagrammatic expansion while the latter is the mean field theory. 

Next, let us seek an (semi-)analytical treatment. A useful approximation is $\phi(\tau)=\phi$, \ie time independent. ({\color{red}How to justify this?}) Then, $\int \mathcal{D}\phi\rightarrow\int\md\phi$ and
\bea S(\phi)=2\log(1+\me^{-\beta|\phi|})+2\log(1+\me^{\beta|\phi|})-\frac{\beta|\phi|^2}{g} \eea  

A second approximation is to truncate $k$ with a upper bound $N_k$. When $N_k=0$, we have omitted all closed diagrams, \ie $\sum[{\rm closed}]=1$. Then, we have
\bea G_R=\left. \int\md \phi \frac{1}{2}\left(\frac{1}{i\omega_n-|\phi|}+\frac{1}{i\omega_n+|\phi|}\right) \me^{S(\phi)}  \middle/ \int\md  \phi \me^{S(\phi)}\right. \eea


Next, let's set $N_k=1$.

\bibliography{bibfile}
%\begin{thebibliography}{99}
%	\bibitem{xxxx} xxxx, PRL 88, 888888 (2088).
%\end{thebibliography}
\end{document}
